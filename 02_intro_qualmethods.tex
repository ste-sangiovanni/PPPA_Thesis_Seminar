\documentclass[10pt, aspectratio=169]{beamer}
\usetheme{Copenhagen}
\usecolortheme{seahorse}
\usefonttheme{serif}

\usepackage[utf8]{inputenc}
\usepackage{hyperref}
\usepackage[T1]{fontenc}
\usepackage{graphicx}
\usepackage{caption}
\usepackage{amsmath}
\usepackage{amsfonts}
\usepackage{amssymb}
\usepackage[absolute,overlay]{textpos}
\usepackage{tikz}
\usepackage{multicol}
\usepackage{ragged2e}
\usepackage{caption}
\usepackage{amsmath}
\usetikzlibrary{positioning}

\newcommand{\shorttitle}{Introduction to Qualitative Research}
\newcommand{\shortauthor}{Stefano Sangiovanni}

\setbeamertemplate{navigation symbols}{}
\defbeamertemplate*{footline}{myfootline}{%
  \ifnum\insertframenumber>1
    \hfill\insertframenumber/\inserttotalframenumber
    \hfill
}


\setbeamertemplate{footline}
{
  \leavevmode%
  \hbox{%
  \begin{beamercolorbox}[wd=.4\paperwidth,ht=2.25ex,dp=1ex,center]{author in head/foot}%
    \usebeamerfont{author in head/foot}\shortauthor \hspace*{2em}
  \end{beamercolorbox}%
  \begin{beamercolorbox}[wd=.4\paperwidth,ht=2.25ex,dp=1ex,center]{title in head/foot}%
    \usebeamerfont{title in head/foot}\shorttitle\hspace*{2em}
  \end{beamercolorbox}%
  \begin{beamercolorbox}[wd=.2\paperwidth,ht=2.25ex,dp=1ex,right]{title in head/foot}%
    \insertframenumber\,/\,\inserttotalframenumber\hspace*{1em}
  \end{beamercolorbox}}%
  \vskip0pt
}


%%%%%%%%%%%%%%%%%%%%%%%%%%%%%%%%%% TITLE PAGE
\title{\textbf{Introduction to Qualitative Research}}
\author{Stefano Sangiovanni \\ \vspace{0.1cm} \small stefano.sangiovanni@unimi.it}
\date{Milan, March 1, 2024}
\newcommand{\coursename}{Politics, Philosophy and Public Affairs  \\ Thesis Seminars for PPPA students}
\newcommand{\academicyear}{A.Y. 2023/24}

\begin{document}
\begin{frame}[plain]
    \centering
    \vspace*{1em} 
    \footnotesize \coursename \\
    \setbeamerfont{title}{size=\Large}
    \setbeamerfont{author}{size=\large}
    \titlepage
    \centering
\end{frame}



\begin{frame}{Who am I? :)}
\begin{itemize}
    \item \textbf{PhD candidate in Political Studies} (NASP - UniMi)\vspace{0.3cm}
    \item \textbf{Research interests}: Political parties, intra-party politics, leadership, crises and political scandals\vspace{0.3cm}
    \item \textbf{Methodology}: Quantitative and computational methods (but I also really like qualitative stuff!)\vspace{0.3cm}
    \item \textbf{Background}: Sociology, Administrations and Public Policy \vspace{0.3cm}
\end{itemize}
\end{frame}

\begin{frame}{Index}
\begin{columns}
\column{0.6\textwidth}
\begin{itemize}
\item Introduction to Qualitative Research \vspace{0.3cm}
\item In-depth interviews \vspace{0.3cm}
\item Some advices on how to do interviews\vspace{0.3cm}
\item Ethical Considerations \vspace{0.3cm}
\item Introduction to Mixed Methods \vspace{0.3cm}
\item Q\&A session
\end{itemize}
\column{0.4\textwidth}
\begin{figure}
    \centering
    \includegraphics[width=\linewidth]{Images/thisisfine.png}
    \label{fig:enter-label}
\end{figure}
\end{columns}
\end{frame}

%%%%%%%%%%%%%%%%%%%%%%%%%%%%%%%%%%%%%%%%%%%%%%%% post index content %%%%%%%%%%%%%%%%%%%%%%%%%%%%%%%%%%%%%%%%%%%%%%%%%%%%%%%%%%%%%%%%%%%%%%%%%%%%%%%%%%%%%%%%%%%%%%%%%%%%%%%%%%%%%%%%%%%%%%%%%%%%%%%%%%%%%%%%%%%%%%%%%%%%%%%%%%%%%%%%%%%%%%%%%%%%%%%%%%%%%%%%%%%%%%%%%%%%%%%%%%%%%%%%%%%%%%%%%%%%%%%%%%%%%%%%%%%%%%%%%%%%%

\begin{frame}{Introduction to Qualitative Research}
\begin{itemize}
\item Offers \textbf{in-depth} exploration and understanding of phenomena\vspace{0.3cm}
\item Utilizes an \textbf{interpretative} approach to uncover meanings and contexts\vspace{0.3cm}
\item Involves different forms of data collection\vspace{0.3cm}
\end{itemize}
\vspace{0.3cm}
\textbf{What really sets qualitative apart from quantitative research?}
\begin{itemize}
\item Quantitative research relies on standardized data collection methods\vspace{0.3cm}
\item Qualitative research adopts a flexible, \textbf{context-sensitive approach} tailored to the research context\vspace{0.3cm}
\end{itemize}
\centering
\textbf{Let method be the servant, not the master!}
\end{frame}

\begin{frame}{Pillars of Qualitative Analysis}
\begin{columns}
\column{0.6\textwidth}
\begin{itemize}
    \item \textbf{Subject}: The individual in their entirety and specificity\vspace{0.3cm}
    \item \textbf{Objective}: Understanding subjects/interpreting a phenomenon\vspace{0.3cm}
    \item \textbf{Type of Analysis}: Classifications, "ideal types"\vspace{0.3cm}
    \item \textbf{Collected Data}: Texts from discursive interviews, ethnographies, official documents\vspace{0.3cm}
    \item \textbf{Representativeness}: Limited to individual cases\vspace{0.3cm}
\end{itemize}
\column{0.4\textwidth}
\begin{figure}
    \centering
    \includegraphics[width=0.6\linewidth]{Images/pillars.jpg}
    \label{fig:pillars}
\end{figure}
\end{columns}
\end{frame}


\begin{frame}{Qualitative research as an archipelago of techniques}
\begin{columns}[T]
\column{0.5\textwidth}
Different techniques, according to: \vspace{0.3cm}
\begin{itemize}
    \item Observational approach\vspace{0.3cm}
    \item Phenomenon under investigation\vspace{0.3cm}
    \item Nature of produced empirical documentation
\end{itemize}
\column{0.5\textwidth}
\begin{figure}
    \centering
    \includegraphics[width=0.8\linewidth]{Images/Archipelago.jpg}
    \label{fig:archipelago}
\end{figure}
\end{columns}
\end{frame}

\begin{frame}{Observational approach}
\begin{center}
\textbf{Focus on the agency of the researcher} \\
Is the phenomenon under study "perturbed" by the researcher's actions? 
\end{center}
\begin{figure}
    \centering
    \includegraphics[width=\linewidth]{Images/Perturbation.png}
    \label{fig:enter-label}
\end{figure}
\end{frame}


\begin{frame}{The Strength of Qualitative Research}
\begin{columns}
\column{0.7\textwidth}
\begin{itemize}
    \item \textbf{Context Sensitivity}: Participants can express themselves in their own words, to act in their natural environments\vspace{0.3cm}
    \item \textbf{Openness and Flexibility}: Meeting the unexpected, producing new concepts\vspace{0.3cm}
    \item \textbf{Participants' Cooperation}: Relevant to the issue of invisibility\vspace{0.3cm}
    \item \textbf{Multivocality}: Evoking the emotions of the field
\end{itemize}
\column{0.3\textwidth}
\begin{figure}
    \centering
    \includegraphics[width=0.8\linewidth]{Images/Cardano.jpg}
    \label{fig:qualitative-strength}
\end{figure}
\end{columns}
\end{frame}

%%% INTERVIEWS


\begin{frame}{Two "macro-types" of interviews}
    \begin{columns}[T]
        \begin{column}{0.5\textwidth}
            \textbf{Structured interviews}
            \begin{itemize}
                \item Impersonal data collector\vspace{0.3cm}
                \item Directed question-answer\vspace{0.3cm}
                \item Closed answers\vspace{0.3cm}
            \end{itemize}
        \end{column}
        \begin{column}{0.5\textwidth}
            \textbf{Qualitative interviews}
            \begin{itemize}
                \item In-depth\vspace{0.3cm}
                \item Flexible, with an outline\vspace{0.3cm}
                \item Aimed at understanding informants’ perspectives on their lives \vspace{0.3cm}
                \item Experiences expressed in their own words
            \end{itemize}
        \end{column}
    \end{columns}
\end{frame}

\begin{frame}{Interviews and the Research Design}
Should interviews be included in your research design? \\ \vspace{0.3cm}
The point of a qualitative interview is to let the respondent tell their own story/narrative on their own terms. \\\vspace{0.3cm}
Interviews can be more useful if:
\begin{itemize}
\item Settings and the people are not accessible otherwise (e.g., past events, life histories) \vspace{0.3cm}
\item You are interested in the \textbf{narratives} about a specific issue.
\end{itemize}
\end{frame}

            
\begin{frame}{Useful Tips for in depth interviews}
    \begin{columns}[T]
        \begin{column}{0.6\textwidth}
    \begin{itemize}
                \item Not a questionnaire! Not a survey!\vspace{0.3cm}
                \item The guide acts as a prompt, reminding you of necessary topics to cover and questions to ask \vspace{0.3cm}
                \item Remind yourself of the key topics to be explored\vspace{0.3cm}
                \item Try to always have in mind your outline
    \end{itemize}
    \end{column}
    \begin{column}{0.4\textwidth}
    \begin{figure}
        \centering
        \includegraphics[width=0.8\linewidth]{Images/Guide.jpg}
        \label{fig:enter-label}
    \end{figure}
    \end{column}
    \end{columns}
\end{frame}

\begin{frame}{Getting People to Talk with You}
    Set the right tone at the beginning of the interview.
\vspace{0.3cm}
    Descriptive questions such as:
    \begin{itemize}
        \item "Can you tell me about a typical day in your life?"\vspace{0.3cm}
        \item "I am interested in X; could you tell me how you got involved?"\vspace{0.3cm}
        \item "I would like to know about dates/people/events/places that are most important to you. Could you start by listing them?"\vspace{0.3cm}
        \item "Can you walk me through that experience?"
    \end{itemize}
\end{frame}

\begin{frame}{How to keep the "engagement"}
            \begin{itemize}
                \item Ask for clarifications\vspace{0.3cm}
                \item Seek more details about feelings, places, and people described by the participant\vspace{0.3cm}
                \item Bring back memories\vspace{0.3cm}
                \item Engage in "knowing in the making"\vspace{0.3cm}
                \item Try to cross-check for inconsistencies or deception\vspace{0.3cm}
            \end{itemize}    
\end{frame}

\begin{frame}{Listening and support techniques}
\begin{columns}
\column{0.7\textwidth}
    \begin{itemize}
        \item \textbf{Silence}: moments of pause allow individuals to process their thoughts\vspace{0.3cm}
        \item \textbf{Continuators}: non-verbal signals indicating attention and interest in listening\vspace{0.3cm}
        \item \textbf{Echo Technique}: pepetition of keywords or phrases to confirm understanding and support the speaker\vspace{0.3cm}
        \item \textbf{Recapitulation}
    \end{itemize}
\column{0.3\textwidth}
\begin{figure}
    \centering
    \includegraphics[width=0.8\linewidth]{Images/listening.jpg}
    \label{fig:enter-label}
\end{figure}
\end{columns}
\end{frame}


\begin{frame}{Ethics of (qualitative) Research}
\begin{columns}
\column{0.5\textwidth}
Key principles:
\begin{itemize}
    \item Minimization of harm\vspace{0.3cm}
    \item Respect for autonomy\vspace{0.3cm}
    \item Protection of privacy and confidentiality\vspace{0.3cm}
    \item Be always careful to research/subject relation\vspace{0.3cm}
    \item Ethics in research: Mertonian Norms of Science
\end{itemize}
\column{0.5\textwidth}
\begin{figure}
    \centering
    \includegraphics[width=0.8\linewidth]{Images/Ethics.jpg}
    \label{fig:ethical-considerations}
\end{figure}
\end{columns}
\end{frame}

\begin{frame}{Research and the Invasion of Privacy}
\begin{center}
Researchers must get \textbf{informed consent} from subjects \\    
\end{center}
\vspace{0.3cm}
Important aspects of the informed consent:    
    \begin{itemize}
        \item two (signed) copies\vspace{0.3cm}
        \item who you are, what is your research about\vspace{0.3cm}
        \item they can interrupt the interview in any copy\vspace{0.3cm}
        \item the records will be destroyed after X years\vspace{0.3cm}
    \end{itemize}
\begin{center}
    \textbf{Always protect and be careful to anonymity and confidentiality}
\end{center}
\end{frame}

\begin{frame}{An Introduction to Mixed Methods}
    \begin{columns}
        \begin{column}{0.6\textwidth}
    \begin{itemize}
        \item Perhaps "mixed methods" are the best of both worlds…\vspace{0.3cm}
        \item Combining quantitative and qualitative elements\vspace{0.3cm}
        \item Qualitative - Quantitative design\vspace{0.3cm}
        \item Quantitative - Qualitative design
    \end{itemize}
    \end{column}
    \begin{column}{0.4\textwidth}
    \begin{figure}
    \centering
    \includegraphics[width=0.8\linewidth]{Images/mix.png}
    \label{fig:ethical-considerations}
\end{figure}
    \end{column}
    \end{columns}
\end{frame}

\begin{frame}{Motivations for Mixed Methods}
    \begin{itemize}
        \item Qualitative as a follow-up to (e.g. interviews) to quantitative findings\vspace{0.3cm}
        \item Exploratory qualitative study (e.g. identify hypotheses) and then quantitative study for empirical testing\vspace{0.3cm}
    \end{itemize}
\centering
\textbf{Always justify your research design! }
\begin{figure}
    \centering
    \includegraphics[width=0.6\linewidth]{Images/mixedmethods.PNG}
    \label{fig:enter-label}
\end{figure}
\end{frame}

\begin{frame}{Challenges of Mixed Methods}
\begin{columns}
\column{0.5\textwidth}
\begin{itemize}
    \item Mismatching of indicators from qualitative and quantitative components\vspace{0.3cm}
    \item Feasibility: more time, work and skills!\vspace{0.3cm}
    \item It's difficult to see (and to justify!) how two parts fit into the whole\vspace{0.3cm}
\end{itemize}
\column{0.5\textwidth}
\begin{figure}
    \centering
    \includegraphics[width=0.8\linewidth]{Images/VALUE.png}
    \label{fig:enter-label}
\end{figure}
\end{columns}
\vspace{0.5cm}
\centering
Use mixed methods only if adds \textbf{value} to your research!
\end{frame}




%%%%%%%%%%%%%%%%%%%%%%%%%%%%%%%%%%%%%%%%%%%%%%%%%%%%%%%%%%%%%%%%%%%%%%%%%%%%%%%%%%%%%%%%%%%%%%%%%%%%%%%%%%%%%%%%%%%%%%%%%%%%%%%%%%%%%%%%%%%%%%%%%%%%%%% Final slide %%%%%%%%%%%%%%%%%%%%%%%%%%%%%%%%%%%%%%%%%%%%%%%%%%%%%%%%%%%%%%%%%%%%%%%%%%%%%%%%%%%%%%%%%%%%%%%%%%%%%%%%%%%%%%%%%%%%%%%%%%%%%%%%%%%%%%%%%%%%%%%%%%%
\begin{frame}[plain]
\centering
\vspace{0.6cm}
\textbf{\large Thank You for Your Attention! Any questions?}
\vspace{0.3cm}
\begin{figure}
    \centering
    \includegraphics[width=0.8\linewidth]{Images/Galadriel.jpg}
    \label{fig:enter-label}
\end{figure}
\end{frame}


%%%%%%%%%%%%%%%%%%%%%%%%%%%%%%%%%%%%%%%%%%%%%%%%%% BIBLIOGRAPHY %%%%%%%%%%%%%%%%%%%%%%%%%%%%%%%%%%%%%%%%%%%%%%%%%%%%%%%%%%%%%%%%%%%%%%%%%%%%%%%%%%%%%%%%%%%%%%%%%%%%%%%%%%%%%%%%%%%%%%%%%%%%%%%%%%%%%%%%%%%%%%%%%%%%%%%%%%%%%%%%%%%%%%%%%%%%%%%%%%%%%%%%%%%%%%%%%%%%%%%%%%%%
\begin{frame}{Bibliography}

\begin{itemize}
    \item Cardano M. (2020), Defending Qualitative Research: Design, Analysis and Textualization, London, Routledge
    \item Hammersley and Traianou, Ethics in Qualitative Research 2012
    \item Morgan, D.L. (1996), Focus groups. Annual review of sociology, 22.1:129-152.
    \item Morgan, D. L., Krueger, R. A., \& King, J. A. (1998). The focus group kit, Vols. 1–6. Sage.
    \item Patton, M. Q. (2005). Qualitative Research. In Encyclopedia of Statistics in Behavioral Science. Wiley. 
\end{itemize}
\end{frame}



\end{document}